\section{Example Application (actin)}


In this section, the construction of a model of an actin molecule is described
starting with a single actin.

\subsection{Monomer construction}

\begin{singlespace}
\ \\
------------------------------------------
actin.run
------------------------------------------
\begin{verbatim}
PARAM actin.model
END
GROUP 0 1
INPUT actin.dat
END
\end{verbatim}
---------------------------------------------------------------------------------------------------
\end{singlespace}

The parameter file {\tt actin.model} specifies a 4-level hierarchy which is almost identical to
the example given previously.

\begin{singlespace}
\ \\
------------------------------------------
actin.model
---------------------------------------
\begin{footnotesize}
\begin{verbatim}
PROT sse 
   0,    3,    3,    2,    1,    0,    0,    0,    0,    0,  	// type
3000, -150,  -70,   -8,    2,    0,    0,    0,    0,    0,  	// size
   0,  150,   70,    8,    8,    0,    0,    0,    0,    0,  	// bump
   0,    0,    0,    6,    4,    0,    0,    0,    0,    0,  	// links
   0,    0,    3,    1,    1,    0,    0,    0,    0,    0,  	// bonds
   0,    1,   -1,   -1,    1,    0,    0,    0,    0,    0,  	// move
   0,    1,    1,    1,    0,    0,    0,    0,    0,    0,  	// keep
   0,    0,    0,    0,    4,    0,    0,    0,    0,    0,  	// bond 
   0,   10,   10,   10,   10,    0,    0,    0,    0,    0,  	// hard
   0,    1,    2,    5,    0,    0,    0,    0,    0,    0,  	// soft
\end{verbatim}
\end{footnotesize}
---------------------------------------------------------------------------------------------------
\end{singlespace}

The {\tt INPUT} file {\tt actin.dat} reads each of actin's four domains in separate files
into an oblate ellipsoid and reconnects the chain through the domains using {\tt REBOND}
commands.

\begin{singlespace}
\ \\
------------------------------------------
actin.dat
------------------------------------------
\begin{verbatim}
GROUP 5 4    -5.0  1.0  0.0    5.0  0.0  0.0
INPUT actin.dom1.dat
INPUT actin.dom2.dat
INPUT actin.dom3.dat
INPUT actin.dom4.dat
REBOND atomid  32 127
REBOND atomid 172  33
REBOND atomid  90 173
REBOND atomid 216 281
REBOND atomid 372 217
REBOND atomid 280  91
RETERM atomid   1 126
\end{verbatim}
---------------------------------------------------------------------------------------------------
\end{singlespace}

Each of the four domains are also contained within an ellipsoid and the top part
of the first two show how the ellipsoids are tailored to each domain shape.
The division of the {\tt ATOM} records into secondary structure groups is generated
automatically, along with the axis end-points by a program that is described in an appendix.

\begin{singlespace}
\ \\
------------------------------------------
actin.dom1.dat
---------------------------------------
\begin{verbatim}
GROUP 10 19    0  1  0    0 -1  0
GROUP 0 6
ATOM      1  CA  GLY A   1     -23.868 -17.022  -6.339  1.33  1.00
ATOM      2  CA  GLY A   2     -21.762 -20.183  -6.100  1.33  1.00
ATOM      3  CA  GLY A   3     -18.845 -17.924  -6.394  1.33  1.00
ATOM      4  CA  GLY A   4     -20.602 -14.455  -6.961  1.33  1.00
ATOM      5  CA  GLY A   5     -20.699 -14.221  -3.138  1.33  1.00
ATOM      6  CA  GLY A   6     -17.825 -16.538  -2.703  1.33  1.00
GROUP 2 5   -14.79 -14.10 -3.15   -4.36 -8.33 -0.83
ATOM      7  CA  GLY A   7     -14.841 -14.717  -4.363  1.33  2.00
ATOM      8  CA  GLY A   8     -12.300 -13.449  -1.827  1.33  2.00
ATOM      9  CA  GLY A   9      -9.949 -10.497  -2.418  1.33  2.00
ATOM     10  CA  GLY A  10      -6.767 -10.364  -0.491  1.33  2.00
ATOM     11  CA  GLY A  11      -4.370  -7.485  -0.937  1.33  2.00
GROUP 0 6
ATOM     12  CA  GLY A  12      -1.017  -8.244   0.718  1.33  1.00
ATOM     13  CA  GLY A  13       0.995  -5.192   1.721  1.33  1.00
:
120 more lines
:
ATOM    119  CA  GLY A 119      -0.359 -27.580 -16.028  1.33  1.00
ATOM    120  CA  GLY A 120       1.825 -26.970 -12.925  1.33  1.00
GROUP 1 6    1.77 -24.25 -13.35    0.40 -17.76 -15.53
ATOM    121  CA  GLY A 121       2.324 -23.445 -11.325  1.33  3.00
ATOM    122  CA  GLY A 122       3.335 -22.259 -14.760  1.33  3.00
ATOM    123  CA  GLY A 123      -0.391 -22.075 -15.685  1.33  3.00
ATOM    124  CA  GLY A 124      -1.234 -18.947 -13.632  1.33  3.00
ATOM    125  CA  GLY A 125       2.182 -17.434 -14.080  1.33  3.00
ATOM    126  CA  GLY A 126       1.159 -16.277 -17.640  1.33  3.00
SHEET
BETA   9 20 7
BETA   20 9 7
BETA   9 58 6
BETA   8 57 6
BETA   58 9 6
:
40 more lines
:
BETA   19 29 1
BETA   17 30 1
BETA   17 12 1
BETA   16 32 1
BETA   12 17 1
\end{verbatim}
---------------------------------------------------------------------------------------------------
\end{singlespace}

The first domain is contained in a sphere ({\tt sort = 10}) but because it is an ellipsoidal object
it has axis end-points defined.  Since the value of {\tt sort} is positive, these end-points only
specify the direction of the axis and not its length.

The second domain has a negative {\tt sort} value so the length of the specified axis sets the size.
As the absolute value of {\tt sort} is over 10, the object will be a prolate ellipsoid with a 
diameter scaled to preserve the axial ratio associated with {\tt sort} = 15.

\begin{singlespace}
\ \\
------------------------------------------
actin.dom2.dat
---------------------------------------
\begin{verbatim}
GROUP -15 8     0 0 -25   0 -10 25
GROUP 2 5    1.42 -7.71  8.19    6.31 -14.04 18.22
ATOM      1  CA  GLY A   1       1.363  -7.636   8.264  1.00  2.00
ATOM      2  CA  GLY A   2       3.361  -8.978  11.218  1.00  2.00
ATOM      3  CA  GLY A   3       3.992 -12.395  12.769  1.00  2.00
ATOM      4  CA  GLY A   4       5.160 -12.308  16.329  1.00  2.00
ATOM      5  CA  GLY A   5       7.149 -15.046  17.922  1.00  2.00
GROUP 0 13
ATOM      6  CA  GLY A   6       7.323 -15.011  21.823  1.00  1.00
ATOM      7  CA  GLY A   7      10.702 -14.236  23.284  1.00  1.00
ATOM      8  CA  GLY A   8       9.826 -16.218  26.547  1.00  1.00
:
many more lines
:
\end{verbatim}
---------------------------------------------------------------------------------------------------
\end{singlespace}

The full path of the \CA-chain through the four domains is: 1-2-1-3-4-3-1, which corresponds to a
branched (Y-shaped) topology with domain 1 being visited three times.  To accommodate this, the
value of $nbonds = 3$ in the parameter file for the domain level.   Strictly, all the others should
be 2, but the program allocates a minimum of 2 anyway if a chain is requested.   In a linear chain 
these are referred to by the 'short-hand' pointers {\tt sis} for the previous object in the chain
and {\tt bro} for the following.   Thus a chain of 4 objects (A--D) appears as:
\begin{singlespace}
\begin{verbatim}
                     bro-->B    bro-->C    bro-->D    bro-->x
                      A          B          C          D
                 x<--sis    A<--sis    B<--sis    C<--sis
\end{verbatim}
\end{singlespace}
With {\tt x} being an end-of-chain marker (actually the parent cell).  If the chain is
circular then {\tt D.bro-->A} and {\tt A.sis-->D}.
In terms of the underlying structure that holds the bonding data, is equivalent to:
\begin{singlespace}
\begin{verbatim}
                    bond1-->B  bond1-->C  bond1-->D  bond1-->(x)
                    next1 = 1  next1 = 1  next1 = 1  next1 = -1
                      A          B          C          D
               -1 = next0  0 = next0  0 = next0  0 = next0
              (x)<--bond0  A<--bond0  B<--bond0  C<--bond0
\end{verbatim}
\end{singlespace}
The seemingly redundant {\tt next} values tell where to find the next link.
However, in a branched topology they have a less trivial job as can be seen when the {\tt C}
object is made into a side branch:
\begin{singlespace}
\begin{verbatim}
                                  bond1-->(x)
                                  next1 = -1
                                    C      
                               2 = next0 
                               B<--bond0    
                                                 
                                                 
                                                 
                                  bond2-->D      
                                  next2 = 0      
                                    B            
                    bond1-->B     bond1-->C     bond1-->(x)
                    next1 = 1     next1 = 0     next1 = -1
                      A             B             D    
                -1 = next0     0 = next0     0 = next0
               (x)<--bond0     A<--bond0     B<--bond0
\end{verbatim}
\end{singlespace}
So following the path of links from {\tt A1}:
\begin{verbatim}
        A1-->B1-->C0-->B2-->D0-->B0-->A0-->(x)
\end{verbatim}
In this representation, the connection through the actin domains is:
\begin{verbatim}
        A1-->B0-->A2-->C1-->D0-->C0-->A0-->(x)
\end{verbatim}


\subsection{Filament construction}

Actin polymerises into a helical filament (with a left-handed twist) which consists of
repeated actin molecules related by a rotation of -166\degre\ and 27.5\AA\ translation
\cite{HolmesKC09}.  The subunit used here is taken from the structure of the filament
(PDB code = {\tt 1M8Q}) which has a fiber axis along Z, the simple {\tt HELIX} command
(described in the previous section) can be used to regenerate the filament.  However,
as the internal coordinates of the molecule will have been centred on the origin,
these require the first two values of the {\tt HELIX} command to shift the molecule
(in X and Y) off the axis, then the standard helical parameters quoted above can be
applied.   The file {\tt actin13.run} generates 13 subunits which is the half-repeat
length of the filament.  For longer filaments, clearly it would be worthwhile writing
a script to generate the run-file.

\begin{singlespace}
\ \\
------------------------------------------
actin13.run
------------------------------------------
\begin{verbatim}
NOMOVE
HIDDEN
PARAM actin.model
END
GROUP 0 13
HELIX -16  -5    0.0     0
INPUT actin.dat
HELIX -16  -5  -27.5  -166
INPUT actin.dat
HELIX -16  -5  -55.0  -332
INPUT actin.dat
HELIX -16  -5  -82.5  -498
INPUT actin.dat
HELIX -16  -5 -110.0  -664
INPUT actin.dat
HELIX -16  -5 -137.5  -830
INPUT actin.dat
HELIX -16  -5 -165.0  -996
INPUT actin.dat
HELIX -16  -5 -192.5 -1162
INPUT actin.dat
HELIX -16  -5 -220.0 -1328
INPUT actin.dat
HELIX -16  -5 -247.5 -1494
INPUT actin.dat
HELIX -16  -5 -275.0 -1660
INPUT actin.dat
HELIX -16  -5 -302.5 -1826
INPUT actin.dat
HELIX -16  -5 -330.0 -1992
INPUT actin.dat
END
\end{verbatim}
---------------------------------------------------------------------------------------------------
\end{singlespace}
