\section{Example Application}

The program described in the previous section corresponds to a general
geometry engine but does not contain any implementation of forces that
are more typically found in molecular dynamics (MD) or Monte Carlo (MC)
programs.  As it stands, the objects simulated will simply shake and diffuse
in a random way under the constraints of the imposed steric exclusion,
linkage constraints and shape envelope confinement.  For anything of
biological interest to occur,  such as a transition from one state to
another, then high level directions must be imposed.   These can range
from being completely determined, where the starting and final states
are fully specified and only the pathway between is of interest or
the change can be applied to just a part of the system and the interest
is to see how the full system adjusts.

To illustrate how such a system can be implemented in more concrete
terms, we will take the example of a molecular motor, specifically
myosin-V on an actin filament \cite{MehtaADet99,DeLaCruzEMet99,ForkeyJNet03}, in which large directed changes are
applied to a component of the system (myosin)  to drive it from one
defined (bound) state to another then back to the original state but
bound to a different actin molecule along the actin filament \cite{KoderaNet10}.  The 
system will be introduced in two parts: firstly, by describing how the
model was set-up as a hierarchical data structure, then secondly, how
the dynamics were introduced.   The first part is done through data
files that have been alluded to in the previous section, however, the
second part requires direct run-time interaction with the simulation
and this is implemented as a specific user-defined routine called
the {\tt driver} (Figure 1) which interacts only with the common data
structure and executes as an independent parallel process.

\subsection{Model Construction}

We will introduce the data-structure from the top down, starting 
with file ({\tt actmyo.run}) that specifies the two main components:
a myosin-V dimer and the actin filament.
\begin{singlespace}
------------------------------------------
actmyo.run
------------------------------------------
\begin{verbatim}
PARAM myosin.model
PARAM actin.model
END
GROUP 2
MODEL 0
INPUT myosin.dimer.dat
MODEL 1
INPUT actin.linear.dat
\end{verbatim}
\ \ ------------------------------------------------------------------------------------------------------
\end{singlespace}

The above run file directs the INPUT from two files for the myosin dimer
({\tt myosin.dimer.dat}) and an actin filament ({\tt actin.linear.dat})
that constitute two units (GROUPs) at the highest level.  Each group 
is preceded by a specification of the parameter set (MODEL) that they
will use (0 and 1) which corresponds to the PARAMeter files:
{\tt myosin.model} and {\tt actin.model}, respectively.  These files consist
of columns of numbers with each column specifying the values for the
different parameters at each level in the hierarchy.   The file 
{\tt myosin.model} consists of seven columns:

\begin{singlespace}
-----------------------------------------
myosin.model
-----------------------------------------
\begin{verbatim}
	   0,    0,    0,    3,    1,    2,    1
	9999, 1000,  500,  140,   70,   24,    5
	   0,  100,  100,   50,   20,   20,   11
	   0,    0,    0,    1,    2,    6,    4
	   0,    0,    0,    1,    3,    1,    1
	   0,    0,    1,    1,    1,    1,    1
	   0,    1,    1,    1,    1,    1,    0
	   0,    0,    1,    5,    5,    5,    5
	   0,    0,    0,    1,    0,    0,    3
\end{verbatim}
\ \ ------------------------------------------------------------------------------------------------------
\end{singlespace}

where the first is the state of the 'world' and the following six define
levels in the hierarchy.  For example; the top line specifies the shape
associated with each level with: 1=sphere, 2=cylinder, 3=ellipsoid and 
0 being a virtual sphere that is not rendered.  The following two lines
specify the size of each object and its bumping radius (both in arbitrary
units).   The equivalent lines for the {\tt actin.model} file show a
slightly different structure using different values:

\begin{singlespace}
--------------------------------------
actin.model (part)
--------------------------------------
\begin{verbatim}
	   0,    0,    1,    3,    1,    2,    1
	9999, 3000,  120,  120,   70,   24,    5
	   0,  400,  100,    0,   70,   20,   11
           :
\end{verbatim}
\ \ ------------------------------------------------------------------------------------------------------
\end{singlespace}

\subsubsection{Myosin}

Considering firstly the myosin model, the file {\tt myosin.dimer.dat} contains
another two GROUPs both specified by the file {\tt myosin.dat}:

\begin{singlespace}
--------------------------------------
myosin.dimer.dat
--------------------------------------
\begin{verbatim}
GROUP 2
SPINY 180.0
TRANS   55.0 -250.0 -450.0
INPUT myosin.dat
TRANS  -55.0 -250.0 -450.0
INPUT myosin.dat
\end{verbatim}
\ \ ------------------------------------------------------------------------------------------------------
\end{singlespace}

The first INPUT is preceded by two geometrical transforms in which the contents
of the file are rotated 180\degre\ about the Y axis (SPINY) then translated 
(TRANS x y z).  The second file is only translated but in the opposite direction
along X producing two copies related by a twofold axis, as seen in the X-ray
crystal structure (PDB code: 2DFS) \cite{LiuJet06a}. (Figure 2).

\begin{figure}
\centering
\subfigure[]{
\label{Fig:secstr2DFS}
\epsfxsize=195pt \epsfbox{figs/secstr2DFS.eps}
}
\subfigure[]{
\label{Fig:chains2DFS}
\epsfxsize=195pt \epsfbox{figs/chains2DFS.eps}
}
\subfigure[]{
\label{Fig:myo5full}
\epsfxsize=195pt \epsfbox{figs/myo5full.rod.eps}
}
\subfigure[]{
\label{Fig:myo5head}
\epsfxsize=195pt \epsfbox{figs/myo5head.rod.eps}
}
\caption{
\label{Fig:myo2DFS}
{\bf Myosin-V structure.}
The structure of dimeric myosin-V ({\tt 2DFS}), determined by single-particle cryo-electron microscopy
is shown ($a$) with secondary structures represented in cartoon style (using {\small RASMOL}) with
\AH\ coloured pink and \Bs s yellow.  ($b$) The same structure is shown as a virtual \CA\ 
backbone with the two heavy chains coloured cyan and green and their associated light-chains
in alternating yellow/orange and red/magenta, respectively.
Secondary structure line-segments (``sticks'') are shown as green tubes
for \Bs s and thicker red tubes for \AHs .
($c$) For the full structure of heavy and light chains (excluding the coiled-coil C-terminus) and
($d$) for the globular foot domain.  (The amino-terminus lies in the all-\B\ SH3 domain to the right)
}
\end{figure}

The file {\tt myosin.dat} contains two components of the myosin molecule which 
are referred to as the "head" and the "tail" and are set up as two GROUPs in what
is now the third level in the hierarch.   From the parameter file {\tt myosin.model}
it can be seen (col:4) that these objects are the first to have a defined shape
and will behave and be rendered as ellipsoids.

\begin{singlespace}
------------------------------------------
myosin.dat
------------------------------------------
\begin{verbatim}
GROUP 2
TRANS  -31.0 136.5 59.0
INPUT myosin.head.dat
INPUT myosin.tail.dat
\end{verbatim}
\ \ ------------------------------------------------------------------------------------------------------
\end{singlespace}

The file {\tt myosin.head.dat} specifies the structure of the myosin head-group 
which is the globular kinase domain of the protein while the file {\tt myosin.tail.dat}
specifies the extended alpha-helical tail with its associated calmodulin-like
light chains. (Sometimes referred to as the lever-arm or the "leg").   Although these
files contain coordinate data from the same protein structure ({\tt 2DFS}), they have 
been processed separately which put their centroid to the origin and a translation
(TRANS) was applied to the head group to restore their proper relative positions.
By specifying the head and the tail as separate groups (units), they can be
operated on independently by geometric transforms allowing the {\tt driver} routine
to recreate a large relative motion between them called the "power-stroke" in which
the tail swings through a large angle. (Discussed below).

The myosin head group was split into seven domains, as described previously \cite{TaylorWRet10b},
all of which are contained in distinct files: {\tt head.dom[1-7].dat}.  The level-3 unit that
contains the domains was defined as an ellipsoid and its end-points are defined on the
GROUP definition line (along with the number of children it contains). 

\begin{singlespace}
---------------------------------------
myosin.head.dat
---------------------------------------
\begin{verbatim}
GROUP 7  -1.2 -42.6 -37.2    4.32 34.44 39.0
INPUT head.dom1.dat
INPUT head.dom2.dat
INPUT head.dom3.dat
INPUT head.dom4.dat
INPUT head.dom5.dat
INPUT head.dom6.dat
INPUT head.dom7.dat
REBOND 134 169
REBOND 275 320
REBOND 411 429
REBOND 495 412
REBOND 428 276
REBOND 292 541
REBOND 667 496
REBOND 540 293
REBOND 319 135
REBOND 168 668
\end{verbatim}
\ \ ------------------------------------------------------------------------------------------------------
\end{singlespace}

\begin{figure}
\centering
\subfigure[]{
\label{Fig:domhead}
\epsfxsize=125pt \epsfbox{figs/myo5head.dom.ps}
}
\subfigure[]{
\label{Fig:headslice1}
\epsfxsize=125pt \epsfbox{figs/headslice1.eps}
}
\subfigure[]{
\label{Fig:headslice2}
\epsfxsize=125pt \epsfbox{figs/headslice2.eps}
}
\caption{
\label{Fig:secdoms}
($a$) Secondary structure elements (SSEs) allocated to produce seven domains
in the myosin head structure.
($b$) The SSEs are redrawn with space-filling
spheres placed on each SSE end-point and sliced through by a plane that would also 
contain the tail (or leg) structure.
($c$) As in part $b$ but with domain boundaries sketched and domain codes added:
A1--2 (ankle), B1--2 (binding) and C1--3 (core).   The direction of the X and Y axes
of the internal reference frame are indicated by arrows.
}
\end{figure}

The path of the chain through the seven head-group domains does not correspond simply
to the domain order.  Since each domain group is defined automatically by compactness,
(giving sets of units that should move together), it is necessary to specify the 
chain path through the domains.  This is done using the identity of units
at the atomic level, sequentially numbered over the scope of the current group. 
Each rewiring of the units is specified by a REBOND command which states that the
unit with the first identity number should be bonded to the unit with the second 
identity number.  For example; "REBOND 134 169" specifies that residue 134 should
now link to residue 169.  The resulting loose end at 135 is picked-up by the later
command "REBOND 319 135".   These connections can be specified "by-hand" but are
written automatically by the preprocessing program that defines the domains.
(\Fig{domstrace}).

\begin{figure}
\centering
\subfigure[]{
\label{Fig:headtrace}
\epsfxsize=180pt \epsfbox{figs/headtrace.eps}
}
\subfigure[]{
\label{Fig:actintrace}
\epsfxsize=210pt \epsfbox{figs/actintrace.eps}
}
\caption{
\label{Fig:domstrace}
($a$) The path of the chain through the seven myosin head domains (grey segments as
in Figure 3) is indicated by a black line with an arrow on the carboxy terminus 
(where the chain enters the tail segment).
($b$)  The path of the chain through the four actin domains is shown in the same way.
Although much simpler than the myosin, the chain still creates a branched (Y-shaped)
topology at the domain level.
}
\end{figure}

Each domain file now takes us down to the lowest (atomic) level at which the actual
X,Y,Z, coordinates are encountered.   To avoid the proliferation of many small files,
the two lowest levels (secondary structures and residues) are defined together with
the first GROUP command stating that there are nine secondary structures in the 
group and the second (lower level) GROUP command specifying six ATOMs (atomic-level
units) in each subgroup.   The coordinate data (from the PDB file) is given at the
atom level along with the secondary structure state encoded in the final column as:
1=loop, 2=beta and 3=alpha.   As the secondary structures are defined in the
parameter file as cylinders, they are given end-point coordinates on the GROUP
command line.  If these are omitted, as with the first loop segment, default 
end-points are generated inside the program.

\begin{singlespace}
---------------------------------------
head.dom1.dat
---------------------------------------
\begin{verbatim}
GROUP 9
GROUP 6 
ATOM      1  CA  GLY A   1      -3.537 -37.198 -20.771  1.00  1.00
ATOM      2  CA  GLY A   2      -1.470 -34.014 -20.628  1.00  1.00
ATOM      3  CA  GLY A   3       1.589 -35.231 -18.657  1.00  1.00
ATOM      4  CA  GLY A   4       3.165 -37.036 -21.663  1.00  1.00
ATOM      5  CA  GLY A   5       6.875 -37.568 -22.449  1.00  1.00
ATOM      6  CA  GLY A   6       8.601 -34.333 -23.626  1.00  1.00
GROUP 7    6.15 -31.57 -21.01   12.09 -24.98 -5.29
ATOM      7  CA  GLY A   7       6.079 -32.117 -21.857  1.00  2.00
ATOM      8  CA  GLY A   8       7.638 -29.779 -19.295  1.00  2.00
ATOM      9  CA  GLY A   9       6.751 -28.831 -15.716  1.00  2.00
ATOM     10  CA  GLY A  10       7.995 -26.626 -12.874  1.00  2.00
:
SHEET
BETA   44 36
BETA   43 37
BETA   37 43
BETA   36 44
BETA   36 23
BETA   23 36
\end{verbatim}
\ \ ------------------------------------------------------------------------------------------------------
\end{singlespace}

The links between \Bs s in a SHEET are specified as pairings on the BETA
records at the end of the file.

The structure of the tail component follows along similar lines and will not
be described in detail.

\subsubsection{Actin}

The highest level actin specific file, {\tt actin.linear.dat} describes a segment
of an actin filament which consists of repeated actin molecules related by helical
symmetry with a rotation of -167\degre\ and 55\AA\ translation \cite{HolmesKC09}.  Because of the way
they interact with the myosin, actins were taken in pairs (called a dimer) giving a
shift between dimers of 23\degre\ and 110\AA\ translation\footnote{
NB: the internal coordinates in the data structure are not \AA ngstroms}.  
These relationships could be encoded by separate 
rotate and translate commands but as helical symmetry is common, a combined HELIX
command was created specifying the two components together:

\begin{singlespace}
---------------------------------------
actin.linear.dat
---------------------------------------
\begin{verbatim}
GROUP 16
INPUT actin.dimer.dat
HELIX 0.0 0.0 -5.8 26.0
INPUT actin.dimer.dat
HELIX 0.0 0.0 -11.6 52.0
INPUT actin.dimer.dat
HELIX 0.0 0.0 -17.4 78.0
INPUT actin.dimer.dat
HELIX 0.0 0.0 -23.2 104.0
INPUT actin.dimer.dat
:
\end{verbatim}
\ \ ------------------------------------------------------------------------------------------------------
\end{singlespace}

Each {\tt actin.dimer.dat} file simply introduces another level in the hierarchy
and maintains the same symmetry around the fibre axis but because the dimer centre lies on the
axis, only a shift along Z is needed at the higher level.

\begin{singlespace}
---------------------------------------
actin.dimer.dat
---------------------------------------
\begin{verbatim}
GROUP 2
HELIX  -3.55  -0.45  0.0    0.0
INPUT actin.one.dat
HELIX   3.55   0.45 -2.9 -167.0
INPUT actin.one.dat
\end{verbatim}
\ \ ------------------------------------------------------------------------------------------------------
\end{singlespace}

The actin molecule consists of four domains arranged as shown in \Fig{actintrace},
forming a flat disc which was naturally encoded
as an oblate ellipsoid.   This was (hand) specified by an axis of length 10 (-5 to 5) which
relative to the size of the domain specified in {\tt actin.model} gives an excessive axial ratio 
which is set to the maximum allowed ratio of 5.\\

\begin{singlespace}
-----------------------------------------
actin.one.dat
-----------------------------------------
\begin{verbatim}
GROUP 4    -5.0  0.0  0.0    5.0  0.0  0.0
INPUT actin.dom1.dat
INPUT actin.dom2.dat
INPUT actin.dom3.dat
INPUT actin.dom4.dat
RELINK 32 127
RELINK 172 33
RELINK 90 173
RELINK 216 281
RELINK 372 217
RELINK 280 91
ENDS   372 126
\end{verbatim}
\ \ ------------------------------------------------------------------------------------------------------
\end{singlespace}

As with the myosin head domains, the chain path through the actin domains must be relinked.
(\Fig{actintrace}).
This also entails the creation of a new terminal position which is specified by the ENDS 
command.   (The equivalent myosin command falls in the tail segment).

The domain files ({\tt actin.dom[1-4].dat}) are similar to the equivalent myosin head
group domain files and introduce no novel features.

\subsection{Driver construction}

The {\tt driver} routine encodes the dynamic aspect of the myosin motor.
This includes specifying the actin/myosin bound state along with motion
at the hinge points between the myosin head and tail and the myosin dimer.

\subsubsection{Myosin-V processive "walking"}

The structure of the actin/myosin bound complex is known from electron
microscopy studies \cite{HolmesKCet03,HolmesKCet04} which have been refined by energy minimisation
\cite{HolmesKCet90,ChenLFet02,KawakuboTet05,LiuYet06,LorenzMet10}.
From any given starting configuration, the actin lying closest to the
myosin head was identified and the myosin molecule moved towards it.
When the myosin came within a predefined distance, its orientation was
also refined.   Together these operations "docked" the myosin molecule
into a configuration relative to the actin that corresponded to the
known structure.    When in this position (referred to as ``tightly bound''),
the orientation of the myosin tail relative to the head was rotated
about an axis that corresponded to what can be inferred from the structures
of myosin with tails in different positions.  This motion, known as 
the "power-stroke" can only occur if the other half of the myosin 
dimer is unbound \cite{ReedyMC00,DominguezRet98}.  In this situation, the free myosin will be carried 
along towards a new region of the actin filament where it will then search
for a new binding site.   Cycling through these states leads to a 
processive motion of myosin along the actin filament\footnote{
Note that this processive, or proper walking, motion of myosin-V differs
from that of muscle myosin (II) in which the actin myosin contact is
only transient.  The latter is more akin to a bank of rowers in which
each myosin oar is dipped into the actin river.}.

The mechanics of the myosin walking motion can be decomposed into three
distinct components:  the myosin can be bound or unbound to actin,
the head can be either in a pre- or post- power-stroke position and the
two halves of the myosin dimer can swivel around their dimer interface.
There is no explicit communication between the binding states of the
two myosin molecules, except what can be communicated through their
dimer interface.   This has the form of an alpha-helical coiled-coil
but little is known of how it responds under stress (tension) or how
it affects the diffusion search of an unbound head.   By contrast, the 
power-stroke (PS) transition is well characterised by many structural 
studies and the forward stroke (from pre- to post-PS positions) 
occurs only when the myosin is bound to actin whereas the reverse
stroke (cocking the trigger) occurs in the unbound state.   These
transitions are all driven by an ATP hydrolysis cycle (summarised in
\Fig{cycle}), the details of which will not be discussed here.

\begin{figure}
\centering
\epsfxsize=200pt \epsfbox{figs/cycle.ps}
\caption{
\label{Fig:cycle}
$a$. Both myosins bind ADP and are attached to actin (creating some bending strain from the relaxed
`leg' positions shown in light grey). $b$. With loss of ADP in the left leg, the power stroke can
progress in the right leg.  $c$  On ATP binding, the left leg is released from actin and swings to the
right and with ATP hydrolysis, it to returns to the pre-power-stroke conformation.
(Reproduced from \protect{\cite{ReedyMC00}}, with permission).
}
\end{figure}



The deterministic progression of the myosin walking cycle lends itself
well to being described as a finite state machine in which a transition to
one state opens the options to progress to others.   From a theoretical
viewpoint, it is of interest to investigate how simple this machine can
be to result in processive motion.

\subsubsection{The myosin dimer hinge}

The most independent component of the myosin machine is the dimeric
interface which, by analogy to walking motion, will be referred to as the
'hip' joint.   As little is known about its structure or dynamics, it
was modelled simply as a constraint to hold the ends of the legs at a
fixed distance.

Walking motion requires movement of the legs about the hip-joint and 
rather than rely on the generic diffusion built into the geometry
engine, an additional motion was included in the {\tt driver} routine
to give any unbound myosin a rotational displacement about the hip.
This provides an example of how the {\tt driver} routine can utilise
structural information across levels in the hierarchy as the rotation
is applied to the whole myosin molecule (level-2) whereas the axis
is determined by the domain positions at the end of the tail in 
both molecules (level-4 in separate sub-trees).

While a faster rate of driven motion reduces the search time for the
free myosin head to find a new binding site without disrupting the
position of the bound head, it also gives less time for the generic
collision detection algorithm to avert clashes between the myosin molecules
and the actin filament.  To rectify this, a specific high-level check
was made on the moving myosin based on the distance of the head from the actin
filament (the actin dimer centre) and the other head group.  The tail was
also checked but as this is an elongated substructure, the closest 
approach of the ellipsoid major axes was monitored and if these fell
below a fixed cutoff (set at the same distance as the hip joint) then
the two myosins were separated.

\subsubsection{The actin/myosin binding}

The distance of the myosin head from each actin dimer centre was 
monitored and when this fell within a given range, the myosin was
moved closer to the selected actin.   This choice is made afresh every time
the {\tt driver} routine is activated, so the target actin can vary as the
configuration of molecules changes.  This approach mode, referred to as "loose
binding", which includes no orientation component, continues until a shorter
threshold is reached at which point the closest actin molecule is selected
and the myosin is orientated and translated towards its known binding 
position.   The orientation component is calculated based on the internal
reference frames of the actin and the myosin head by applying the rotation
matrix and translation vector that reproduces the docked complex.
A fully-bound state is declared if this transformation can be applied and
the end of the tail is placed within the range of the hip-joint to the
other myosin. 

\subsubsection{The power-stroke}

The power-stroke (PS) consists of a large swinging motion of the tail relative
to the head which is easily encoded in the {\tt driver} routine as a rotation
about an axis that lies within the head.  Structural studies have associated
the swivel point with a particular \AH\ and the coordinates of this helix
centre were taken as the hub.   The rotation axis is less well defined but can
also be inferred from the known structures and this was calculated and set as
a fixed axis in the {\tt driver} routine.   Similarly, the extent of the left 
and right swing were preset with reference to the internal reference frame of
the myosin head.

Given these constraints, a simple mechanism was encoded that maintained a
position at the end of the swing range depending on the bound state of the 
myosin.   If the myosin was unbound the tail angle was incremented until it
attained the pre-PS position and only when the myosin became fully bound, was
this motion reversed towards the post-PS position.   These two states were
used to impose an additional condition on binding: that the myosin can only
select an actin for binding when it is in the pre-PS position.   This means
that the post-PS state must detach from the actin and revert to the pre-PS
position before it can rebind.  This introduces a refractory period that 
depends on how fast the tail can swing which has a physical correspondence
to the release of ADP and the subsequent binding of ATP.


\subsection{Simulation}

The specification of the {\tt driver} process introduced a number of
parameters that will not be considered in detail here.
These were tested and roughly optimised to
generate processive motion along the actin filament at as high a speed
as could be achieved.   To avoid end effects, the actin filament was bent
into a circular 'race' track and the performance of the system was
monitored by the lap-time for the myosin, combined with a penalty for
myosin runners that fell off the track (i.e., both heads simultaneously
detached).

\begin{figure}
\centering
\subfigure[]{
\epsfxsize=175pt \epsfbox{figs/step1.eps}
}
\subfigure[]{
\epsfxsize=190pt \epsfbox{figs/step2.eps}
}
\subfigure[]{
\epsfxsize=175pt \epsfbox{figs/walkL.eps}
}
\subfigure[]{
\epsfxsize=175pt \epsfbox{figs/walkR.eps}
}
\caption{
\label{Fig:sticks}
$a$. The myosin-V dimer model (red) is shown bound to the actin filament model (green).
Secondary structure elements (SSEs) are depicted as cylinders (with large/small diameters for \A/\B )
connected by fine lines.   The translucent spheres show the higher level groupings of SSEs into domains.
$b$. Annotates part $a$, with the myosin {\bf leading leg} bound tightly to the actin in pre-power-stroke conformation.
The myosin {\bf trailing leg} has just detached from the {\bf actin filamnet} (solid line) and is now free to
pivot about the {\bf hip-joint} between the two legs.  The height ({\bf H}) of the free myosin above the filament
determines when it is recaptured, initially into a loose-binding mode.
Parts $c$ and $d$ capture the myosin in a similar conformation showing the circuit of 104 actin molecules
(52 dimers with 8 half-repeats) over which the myosin can move.
}
\end{figure}

\begin{figure}
\centering
\epsfxsize=190pt \epsfbox{figs/trace.eps}
\caption{
\label{Fig:trace}
Step traces for myosin simulations showing a “staircase” of overlapping "steps"
which is indicative of a brachiating (ape-like) processive motion.
The X-axis is time and the Y-axis actin position in the filament.
Trajectories for the two myosin heads are in black and grey.
Occasional back-steps can be identified, similar to those seen in biophysical studies.
}
\end{figure}
                                                                                            
Although the time scale is arbitrary, this simulation generated
behaviour that was remarkably similar to that observed for real myosin-V
motion by single particle studies (such as realtime AFM \cite{KoderaNet10})
in which the molecules alternatively swing their legs to new binding positions along
the actin filament.   If viewed "upside-down", the motion is remarkably similar
to the brachiating motion of the great apes (especially the gibbon) as they
swing through the trees.  The major divergence from reality in our simulations was
the nature of the rate-limiting step, which in the
real system is the time for release of ADP whereas in our simulation, the
search time for the next binding site was rate limiting.

\begin{figure}
\centering
\subfigure[]{
\epsfxsize=195pt \epsfbox{figs/search.eps}
}
\caption{
\label{Fig:search}
Superposed frames of myosin molecules taken from a simulation.   Each myosin is represented 
in simplified form by a line connecting the mid-points of sequential domains in the tail segment
which is drawn thicker for a tightly bound myosin.  The head groups are represented by a sphere.
The colours represent: active power-stroke (green), post-power-stroke conformation (red), unbound
(blue) and loosely bound (cyan).   The actin filament is shown schematically in black as a 
double helix tracing the two halves of each dimeric unit.  The myosin moves from right to left.
}
\end{figure}
                                                         
Various tests were performed on how rigid the myosin tail (legs) had
to be and how strictly the hip-joint separation had to be maintained.
If the legs were too weak or the hip too wide, then a state could be
attained in which both myosins were tightly bound in the post-PS position.
If this occurred then there was no frustration in the system and both
myosins could remain stably bound forever, resulting in no motion.
At the other extreme, if the legs were too stiff and the hip joint
maintained strictly at a short separation, then the myosin pair did
not have sufficient flexibility to attain simultaneous binding of the
two heads, leading to one head always bound with the other always 
free or in a loosely bound state to various nearby actins --- again
resulting in no motion.
